\chapter{Introdução}
\begin{comment}
Segundo \cite{horn86robot}, todo triângulo equilátero tem os lados iguais. Já
segundo \cite{shashua97photometric}, todo quadrado também tem. \cite{Li2009}

Veja que o pacote \verb|natbib| permite uma série de formas diferentes para
fazer referências bibliográficas. O comando padrão, \verb|\cite|, realiza a
citação comum vista no parágrafo anterior. Outros comandos permitem, por
exemplo, citar somente o autor --- por exemplo, citar o trabalho de
\citeauthor{samaras99coupled} --- ou colocar automaticamente a citação entre
parênteses \citep{hougen93estimation, sato99illumination2, sato99illumination1,
sato01stability}. Os comandos usados foram, respectivamente, \verb|\citeauthor|
e \verb|\citep|. Veja a documentação do \verb|natbib| na Internet para conhecer
outros comandos e exemplos de uso. 

Citações aleatórias para fazer com que as referências bibliográficas ocupem
mais de uma página: \cite{bichsel92simple, dror01statistics, guisser92new}.
\end{comment}

Dado que um banco de dados, temos que um conjunto muito grande de características presentes nele, muitas vezes essas características não são relevantes. A seleção de características é necessário se obter previamente um conjunto de características, através da análise constante de tais características será possível fazer as análises e a predição do estado da máquina.

\section{Motivação e Relevância}

\dummytxtb\dummytxta

\section{Objetivos}

\dummytxtb\dummytxta

\subsection{Objetivos específicos}


\dummytxtc\dummytxtb

\section{Estrutura do Trabalho}

\dummytxta\dummytxtc

\subsubsection{Descendo mais um nível}

\dummytxtb\dummytxta
